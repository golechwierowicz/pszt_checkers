\documentclass{article}
\usepackage{polski}
\usepackage[utf8]{inputenc}
\usepackage{algorithm}
\usepackage{algpseudocode}
\usepackage{indentfirst}
\usepackage{graphicx}

\title{PSZT}
\date{2016-12-21}
\author{Grzegorz Olechwierowicz, Mikołaj Florkiewicz\and Michał Sypetkowski}

\begin{document}
\maketitle
\pagenumbering{gobble}
\newpage
\pagenumbering{arabic}
\section{Problem}

Zadanie polega na stworzenie automatycznego gracza w warcaby opierającego swoją strategię
na algorytmach ewolucyjnych. W celach weryfikacji należy porównać go do podejścia
opartego na MiniMax

\section{Rozwiązanie}

\subsection{Algorytm ewolucyjny}

Jeden wektor w tym algorytmie jest reprezentowany przez tablicę współczynników (IlośćOsiągalnychPozycjiNaPlanszy x 3).
Każda z 3 wartości w tym ostatnim wymiarze odpowiada na pytanie: "Ile dodać do punktacji planszy gdy na danym polu stoi:
[0] - swój pionek
[1] - wrogi pionek
[2] - pole jest puste ?
Algorytm nie uwzględnia damek (nie rozróżnia ich od zwykłych pionków).
Mutacja jest dokonywana losowo, bez zmiennych odchyleń standardowych.

Uczenie będzie zaimplementowane w oddzielnym pliku.
Funkcja celu to winRatio = (ilosćGierWygranych + dodatkowePunkty)/ilośćWszystkichGier
Gdzie dodatkowe punkty liczą się tak:
ilosćRemisów * 0.5
Dodatkowo program umożliwia dodanie flagi "-e", która zmieni funkcję celu, dodając następujące wartości:
+ 0.05 * ilośćPozostałychPionkówPoWygranej
+ 0.15 * ilośćPozostałychDamekPoWygranej
Selekcja - algorytm 1+1 

\subsection{Mini-max}

Jest to standardowy algorytm w którym maksymalizujemy własną, heurystyczną funkcję oceniającą sytuację na planszy, a minimalizujemy przeciwnika. W naszym przypadku funkcja celu jest to różnica ilości własnych pionków od ilości pionków przeciwnika. Algorytm nie implementuje alfa-beta w celu odcinania części drzewa gry. Aktualna implementacja umożliwia przewidywanie 4-5 ruchów w przód, bez większych "zgrzytów" na dzisiejszych komputerach. 

\section{Sprawy techniczne}

Algorytmy są zaimplementowane w języku Python. Dostępne jest również GUI, które pokazuje aktualną grę. W celach porównawczych efektywnościu obu wyżej wymienionych rozwiązań GUI nie jest dostępne.   

\end{document}
